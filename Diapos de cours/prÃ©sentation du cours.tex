\documentclass{beamer}
\usepackage[utf8]{inputenc}   % pour pouvoir taper les accents directement     
\usepackage{amsfonts,amssymb,amsmath}
\usepackage{tikz}
\usepackage{array}
\usetikzlibrary{patterns}
\usepackage[absolute,showboxes,overlay]{textpos}     
\textblockorigin{0pt}{0pt}                          
\TPshowboxesfalse  
 \usepackage{lmodern,multido}

\newcommand{\R}{\mathbb{R}}
\newcommand{\C}{\mathbb{C}}
\newcommand{\Z}{\mathbb{Z}}
\newcommand{\N}{\mathbb{N}}
\newcommand{\Q}{\mathbb{Q}}
\newcommand{\E}{(-4,-1) rectangle (4,4)}
\newcommand{\A}{(0,0) ++(135:2) circle (2)}
\newcommand{\B}{(0,0) ++(45:2) circle (2)}

\begin{document}
 \addtobeamertemplate{navigation symbols}{}{\hspace{1em} \usebeamerfont{footline}%
    \insertframenumber/\inserttotalframenumber }

 %%%%%%%%%%%%%%%%%%%%%%%%%%%%%%%%%%%%%%%%%%%%%%%%%%%%%%%%%%%%%%%


%\item parfois simplifié

\begin{frame}{Introduction aux statistiques 2025-26}{}

\begin{center}{\bf \Large Informations générales} \end{center}
\vspace{0.2cm}

\begin{itemize}

\item Organisation : 9 séances de cours-TD 

\

\item Evaluation : 2 épreuves en présentiel 
\begin{itemize}
\item 1 QCM ($30\%$) + 1 examen terminal ($70\%$) \\
\end{itemize}

\

\item Documents à votre disposition : 
\begin{itemize}
\item distribués en TD : formulaire de cours (autorisé aux examens) + feuilles de TD
\item sur Moodle : diapos + polycopié de cours + corrigés de certains exercices + archives d'examens 
\end{itemize}
\end{itemize}

\end{frame}


\begin{frame}{Introduction aux statistiques 2025-26}{}

\begin{center}{\bf \Large Programme} \end{center}
\vspace{0.2cm}

\begin{itemize}
\item {\color{gray} Rappels sur les probabilités et les variables aléatoires discrètes $\rightarrow$ Diapos, TD et corrigé disponibles sur moodle}

\

\item Variables aléatoires et lois usuelles continues

\

\item Introduction à la théorie de l'estimation 

\

\item Tests statistiques
 \begin{itemize}
 \item Moyennes 
 \item Proportions 
 \item Répartitions
 \end{itemize}
%\item {\it ANOVA et tests non paramétriques non traités} \\
\end{itemize}
\end{frame}

\end{document}

