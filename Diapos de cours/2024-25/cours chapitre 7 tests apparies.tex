\documentclass{beamer}
\usepackage[utf8]{inputenc}   % pour pouvoir taper les accents directement     
\usepackage{amsfonts,amssymb,amsmath}
\usepackage{tikz}
\usepackage{array}
\usepackage{calc}
\usetikzlibrary{patterns}
\usepackage[absolute,showboxes,overlay]{textpos}     
\textblockorigin{0pt}{0pt}                          
\TPshowboxesfalse  
 \usepackage{lmodern,multido}

\newcommand{\R}{\mathbb{R}}
\newcommand{\C}{\mathbb{C}}
\newcommand{\Z}{\mathbb{Z}}
\newcommand{\N}{\mathbb{N}}
\newcommand{\Q}{\mathbb{Q}}

\begin{document}
 %%%%%%%%%%%%%%%%%%%%%%%%%%%%%%%%%%%%%%%%%%%%%%%%%%%%%%%%%%%%%%%
 % Afficher le numéro de diapos 
  \addtobeamertemplate{navigation symbols}{}{ \hspace{1em}    \usebeamerfont{footline}%
    \insertframenumber/\inserttotalframenumber }
 
%%%%%%%%%%%%%%%%%%%%%%%%%%%%%%%%%%%%%%%%%%%%%%%%%%%%%%%%%%%%%%%

\begin{frame}{Introduction aux statistiques}{Données appariées}
\begin{textblock*}{\textwidth}(1cm,2cm)

\begin{center}{\bf \Large Généralités } \end{center}


\

Appariement  = une des deux situations suivantes :
\begin{itemize}
\item  deux expériences sur le même échantillon et comparaison des résultats obtenus
\item expérience sur deux échantillons  non indépendants : les individus  ont des caractéristiques communes.
\end{itemize}



\end{textblock*}

\end{frame}

 
%%%%%%%%%%%%%%%%%%%%%%%%%%%%%%%%%%%%%%%%%%%%%%%%%%%%%%%%%%%%%%%

\begin{frame}{Introduction aux statistiques}{Données appariées}
\begin{textblock*}{\textwidth}(1cm,2cm)

\begin{center}{\bf \Large Comparaison de moyennes } \end{center}


 Exemple :
 
 \begin{itemize}
 \item deux traitements A et B pour diminuer les  infections
opportunes  si infection au VIH
\item durées sans infection opportune  très corrélées
au nombre de lymphocytes CD4
\item couples de patients ayant le même nombre de CD4 
\item attribution aléatoire des traitements A et B dans chaque paire
 \end{itemize}




\end{textblock*}

\end{frame}

 
%%%%%%%%%%%%%%%%%%%%%%%%%%%%%%%%%%%%%%%%%%%%%%%%%%%%%%%%%%%%%%%

\begin{frame}{Introduction aux statistiques}{Données appariées}
\begin{textblock*}{\textwidth}(1cm,2cm)

\begin{center}{\bf \Large Comparaison de moyennes } \end{center}



 
 \begin{itemize}
 \item $X$ (resp. $Y$) durée de vie (en mois) sans infection  des patients traités avec A (resp. B)
\item durées  des couples  notées $(x_1,y_1),\hdots,(x_{100},y_{100})$ 
\item $z_i=x_i-y_i$ :  série statistique $z_1,\hdots,z_{100}$
 \end{itemize}


\begin{center}
\begin{tabular}{|c|c|c|c|c|c|c|c|c|c|c|}
\hline
valeur de $z$ & -4 & -3 & -2 & -1 & 0 & 1 & 2 & 3 & 4 & 6 \\
\hline
nombre de couples & 2 & 4 & 8 & 15 & 22 & 23 & 14 & 8 & 3 & 1 \\
\hline
\end{tabular}
\end{center}

\end{textblock*}

\end{frame}


%%%%%%%%%%%%%%%%%%%%%%%%%%%%%%%%%%%%%%%%%%%%%%%%%%%%%%%%%%%%%%%

\begin{frame}{Introduction aux statistiques}{Données appariées}
\begin{textblock*}{\textwidth}(1cm,2cm)

\begin{center}{\bf \Large Comparaison de moyennes } \end{center}



 
 \begin{itemize}
 \item une seule population (donc un seul échantillon) 
 \item variable  $Z=X-Y$
 \item $E(Z) = \mu_A - \mu_B$ mais  $Var(Z) \not= Var(X) + Var(Y)$ (variables $X$ et $Y$  pas indépendantes)
 \item $\mu=E(Z)$ et $\sigma^2=Var(Z)$
 \item $M$ estimateur de $\mu$ sur des échantillons de taille $n=100$ 
 \item $S^2$ estimateur de  $\sigma^2$ 
 \end{itemize}



\end{textblock*}

\end{frame}

%%%%%%%%%%%%%%%%%%%%%%%%%%%%%%%%%%%%%%%%%%%%%%%%%%%%%%%%%%%%%%%

\begin{frame}{Introduction aux statistiques}{Données appariées}
\begin{textblock*}{\textwidth}(1cm,2cm)

\begin{center}{\bf \Large Comparaison de moyennes } \end{center}




 \begin{itemize}
 \item variable de test  $T=\frac{M-\mu}{S/\sqrt{n}}$
 \item si $n$ est grand ($n\geq 30$), $T\sim \mathcal{N}(0\,;\,1)$
 \item si $X$ et $Y$  distribuées normalement alors $T$ suit une loi de Student à $n-1$ degrés de liberté
 \end{itemize}

Méthode classique :

\begin{itemize}
\item si $(H_0)$ vraie, $T(echantillon) = \frac{\overline{x}-\overline{y}}{\frac{s_{xy}}{\sqrt{n}}}=t\approx 2.29>1.96$ 
\item on  rejette $(H_0)$ :différence significative
\item conclusion $\mu_A > \mu_B$,
\item  degré de signification  0.03.
\end{itemize}

\end{textblock*}

\end{frame}

 
%%%%%%%%%%%%%%%%%%%%%%%%%%%%%%%%%%%%%%%%%%%%%%%%%%%%%%%%%%%%%%%

\begin{frame}{Introduction aux statistiques}{Données appariées}
\begin{textblock*}{\textwidth}(1cm,2cm)

\begin{center}{\bf \Large Comparaison de proportions } \end{center}


 Exemple :
 
 \begin{itemize}
\item deux méthodes pour tenter de dessouler
\item 45 journalistes  même quantité d'alcool 
\item  première soirée, bol de café
\item deuxième soirée,  bol de vinaigre de cidre
\item  nombre de succès.
\item  C+ (resp. C-)  "la technique du café a marché" (resp. "n'a pas marché") 
\item  V+ (resp. V-)  "la technique du vinaigre a marché" (resp. "n'a pas marché")
 \end{itemize}




\end{textblock*}

\end{frame}

 
%%%%%%%%%%%%%%%%%%%%%%%%%%%%%%%%%%%%%%%%%%%%%%%%%%%%%%%%%%%%%%%

\begin{frame}{Introduction aux statistiques}{Données appariées}
\begin{textblock*}{\textwidth}(1cm,2cm)

\begin{center}{\bf \Large Comparaison de proportions } \end{center}



 
\begin{center}
\begin{tabular}{|c|c|c|}
\hline
Résultat Café & Résultat Vinaigre & nombre d'individus \\
\hline
C- & V- & 5 \\
\hline
C- & V+ &  13\\
\hline
C+ & V- &   19\\
\hline 
C+ & V+ &   11 \\
\hline
\end{tabular}
\end{center}

\begin{itemize}
\item population abstraite $\mathcal{P}$ : individus de type 
$(C+,V-)$ ou  $(C-,V+)$ (différence entre les deux traitements)
\item  $\pi$ la proportion (théorique...) d'individus  du type $(C+,V-)$ dans cette population.
\item  $(H_0)\,: \, \pi=0.5$ contre $(H_1)\,: \, \pi\neq 0.5$ 
\end{itemize}
\end{textblock*}

\end{frame}


%%%%%%%%%%%%%%%%%%%%%%%%%%%%%%%%%%%%%%%%%%%%%%%%%%%%%%%%%%%%%%%

\begin{frame}{Introduction aux statistiques}{Données appariées}
\begin{textblock*}{\textwidth}(1cm,2cm)

\begin{center}{\bf \Large Comparaison de proportions } \end{center}



 
 \begin{itemize}
 \item $\mathcal{E}_n$ l'ensemble des échantillons de taille $n$ dans la population $\mathcal{P}$
 \item $ T(echantillon) = \frac{x-y}{\sqrt{n}}$
où $x$ = nombre d'individus  de type $(C+,V-)$ et $y$  nombre d'individus  de type $(C-,V+)$ (rmq : $n=x+y$).
 \item si $n$  grand ($n\geq 30$) alors  $T\sim\mathcal{N}(0\,;\,1)$
 \item $|t|=\left|\frac{13-19}{\sqrt{32}}\right| \approx 1.06<1.96$
 \item Conclusion : pas de différence significative entre les deux méthodes
 \end{itemize}


\end{textblock*}

\end{frame}



%%%%%%%%%%%%%%%%%%%%%%%%%%%%%%%%%%%%%%%%%%%%%%%%%%%%%%%%%%%%%%%

\end{document}

