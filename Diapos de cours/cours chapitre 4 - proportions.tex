\documentclass{beamer}
\usepackage[utf8]{inputenc}   % pour pouvoir taper les accents directement     
\usepackage{amsfonts,amssymb,amsmath}
\usepackage{tikz}
\usepackage{array}
\usepackage{calc}
\usetikzlibrary{patterns}
\usepackage[absolute,showboxes,overlay]{textpos}     
\textblockorigin{0pt}{0pt}                          
\TPshowboxesfalse  
 \usepackage{lmodern,multido}

\newcommand{\R}{\mathbb{R}}
\newcommand{\C}{\mathbb{C}}
\newcommand{\Z}{\mathbb{Z}}
\newcommand{\N}{\mathbb{N}}
\newcommand{\Q}{\mathbb{Q}}

\newcommand{\pop}{\mathcal{U}} % pour ne pas confondre avec la loi de Poisson

\begin{document}
 %%%%%%%%%%%%%%%%%%%%%%%%%%%%%%%%%%%%%%%%%%%%%%%%%%%%%%%%%%%%%%%
 % Afficher le numéro de diapos 
  \addtobeamertemplate{navigation symbols}{}{ \hspace{1em}    \usebeamerfont{footline}%
    \insertframenumber/\inserttotalframenumber }
 %%%%%%%%%%%%%%%%%%%%%%%%%%%%%%%%%%%%%%%%%%%%%%%%%%%%%%%%%%%%%%%

\begin{frame}{Introduction aux statistiques}
\begin{textblock*}{\textwidth}(1cm,2cm)

\begin{center}{\bf \Large Chapitre 5} \end{center}
\begin{center}{\bf \Large Statistique sur proportions} \end{center}
\vspace{0.5cm}
\begin{itemize}
\item Estimation  
\item Intervalle de confiance
\item Tests statistiques d'hypothèses
\end{itemize}
\vspace{0.5cm}
\begin{itemize}
\item Variables quantitatives : moyenne et variance (chapitre 4)
\item {\bf Variables qualitatives : proportion} (chapitre 5)
\end{itemize}

 \end{textblock*}

\end{frame}


 %%%%%%%%%%%%%%%%%%%%%%%%%%%%%%%%%%%%%%%%%%%%%%%%%%%%%%%%%%%%%%

\begin{frame}{Estimation}
\begin{textblock*}{\textwidth}(1cm,2cm)

\begin{center}{\bf \Large Estimation d'une proportion} \end{center}

 $\pi$  proportion d'individus de la population $\pop$ présentant une certaine propriété
 
 \

Estimateur de la proportion : \\
\begin{center}
$\mathrm{P}_n : \mathcal{E}_n \longrightarrow \R$ défini par 
\end{center}

\begin{center}
$\mathrm{P}_n\mbox{(échantillon)} =p=$ proportion dans l'échantillon 
\end{center}

 \begin{itemize}
 \item $n\times \mathrm{P}_n \sim \mathcal{B}(n;\pi)$ approximativement
 \item $E(\mathrm{P}_n)=\pi$ (estimateur sans biais)
 \item $Var(\mathrm{P}_n)=\frac{\pi(1-\pi)}{n}$ (estimateur correct)
 \end{itemize}


\end{textblock*}

\end{frame}


 %%%%%%%%%%%%%%%%%%%%%%%%%%%%%%%%%%%%%%%%%%%%%%%%%%%%%%%%%%%%%%

\begin{frame}{Estimation}
\begin{textblock*}{\textwidth}(1cm,2cm)

\begin{center}{\bf \Large Estimation d'une proportion} \end{center}

\vspace{0.5cm}
De plus, si
\begin{itemize}
\item $n$  grand ($n\geq 30$)
\item $n\pi\geq 5$
\item $n(1-\pi)\geq 5$
\end{itemize} 

\
 $\mathrm{P}_n$ suit approximativement une loi normale 
$\displaystyle \mathcal{N}\left(\pi;\sqrt{\pi(1-\pi)/n}\right)$.

\end{textblock*}

\end{frame}




 %%%%%%%%%%%%%%%%%%%%%%%%%%%%%%%%%%%%%%%%%%%%%%%%%%%%%%%%%%%%%%

\begin{frame}{Intervalle de confiance}
\begin{textblock*}{\textwidth}(1cm,2cm)

\begin{center}{\bf \Large Intervalle de confiance d'une proportion} \end{center}


On pose 
$$
T=\frac{\mathrm{P}_n - \pi}{\sqrt{\frac{\mathrm{P}_n (1-\mathrm{P}_n)}{n}}} \sim \mathcal{N}(0 ; 1) \text{ approximativement}
$$ 
\
à condition que :
\begin{itemize}
\item $n$ grand ($n\geq 30$)
\item $n\pi\geq 5$ ; $n(1-\pi)\geq 5$
\end{itemize}
\vspace{0.3cm}
{\bf Intervalle de confiance à $95 \%$ de $\pi$} :
$$
\left[ p -1.96 \sqrt{\frac{p(1-p)}{n}} ; p +1.96 \sqrt{\frac{p(1-p)}{n}} \right]
$$



 \end{textblock*}

\end{frame}


 %%%%%%%%%%%%%%%%%%%%%%%%%%%%%%%%%%%%%%%%%%%%%%%%%%%%%%%%%%%%%%

\begin{frame}{Intervalle de confiance}
\begin{textblock*}{\textwidth}(1cm,2cm)

\begin{center}{\bf \Large Intervalle de confiance d'une proportion} \end{center}

\vspace{0.5cm}

\noindent {\it Remarque : }  

conditions de validité $n\pi\geq 5$
et $n(1-\pi)\geq 5$ mais $\pi$  inconnu

\
\begin{itemize}
\item remplacer $\pi$ par l'estimation $p$

\

\item calculer d'abord les bornes de l'intervalle de confiance $[\pi_1\,;\,\pi_2]$ et vérifier $n\times\pi_1 \geq  5 \quad n\times (1-\pi_2) \geq  5\,.
$
\end{itemize}

 \end{textblock*}

\end{frame}

%%%%%%%%%%%%%%%%%%%%%%%%%%%%%%%%%%%%%%%%%%%%%%%%%%%%%%%%%%%%%%%

\begin{frame}{Tests statistiques}
\begin{textblock*}{\textwidth}(1cm,2cm)


\begin{center}{\bf \Large Tests de proportions} \end{center}


\

\

\begin{itemize}
\item Conformité
\begin{itemize}
\item Comparaison à une valeur de référence
\end{itemize}

\
\item Homogénéité
\begin{itemize}
\item Comparaison de 2 proportions
\end{itemize}

\
\item Grands échantillons
\end{itemize}


\end{textblock*}

\end{frame}

%%%%%%%%%%%%%%%%%%%%%%%%%%%%%%%%%%%%%%%%%%%%%%%%%%%%%%%%%%%%%%%

\begin{frame}{Test de conformité de proportion}
\begin{textblock*}{\textwidth}(1cm,1.5cm)

\begin{center}{\bf \Large Exemple  } \end{center}


\begin{itemize}
\item  Proportion de bouteilles bouchonnées : $0.1$
\item  500 bouteilles avec bouchons en liège du Portugal
\item  35 bouteilles bouchonnées
\item Qualité du liège portugais ?
\end{itemize}

\
Notations 
\begin{itemize}
\item $\pi$ proportion de bouteilles bouchonnées parmi les bouteilles 
avec un bouchon portugais
\item $P_n$ estimateur de $\pi$ pour des échantillons de taille $n$
\item $p$ estimation de $\pi$ sur l'échantillon observé
\end{itemize}

\end{textblock*}

\end{frame}
  
  
    %%%%%%%%%%%%%%%%%%%%%%%%%%%%%%%%%%%%%%%%%%%%%%%%%%%%%%%%%%%%%%%

\begin{frame}{Test de conformité de proportion}
\begin{textblock*}{\textwidth}(1cm,1cm)

\begin{center}{\bf \Large Principe  } \end{center}

\begin{itemize}
\item Hypothèse testée :
\begin{itemize}
\item  $H_0 : \pi=\pi_0$
\item $H_1 : \pi \neq \pi_0$
\end{itemize}

\

\item Statistique de test : $\displaystyle T=\frac{ P_n-\pi_0}{\sqrt{\frac{\pi_0 (1-\pi_0)}{n}}}$
\begin{itemize}
\item Si  $n\geq 30$, $n\pi\geq 5$ et $n(1-\pi)\geq 5$
$$\displaystyle \frac{ P_n-\pi}{\sqrt{\frac{\pi(1-\pi)}{n}}}
\sim \mathcal{N}(0,1) \text{ approximativement}$$
\item Sous $H_0$ : $\pi=\pi_0$ $\Rightarrow$
$\displaystyle T=\frac{ P_n-\pi_0}{\sqrt{\frac{\pi_0(1-\pi_0)}{n}}} \sim \mathcal{N}(0,1)$ approximativement, \emph{si $n\geq 30$, $n\pi_0\geq 5$ et $n(1-\pi_0)\geq 5$}.
\end{itemize}
\end{itemize}
\end{textblock*}

\end{frame}

  %%%%%%%%%%%%%%%%%%%%%%%%%%%%%%%%%%%%%%%%%%%%%%%%%%%%%%%%%%%%%%%

\begin{frame}{Test de conformité de proportion}
\begin{textblock*}{\textwidth}(1cm,1.5cm)

\begin{center}{\bf \Large Principe  } \end{center}

\begin{itemize}
\item Définition de la zone de rejet de $H_0$
$$
Pr(|T|>z_{\alpha}) = \alpha
$$
\item Calcul de $t$ sur les données de l'échantillon \\

\

\item Régle de décision 
\begin{itemize}
\item Si $|t| > z_{\alpha} \; \Rightarrow \; $ Rejet de $H_0$ \\
$\rightsquigarrow$ Mise en évidence d'une différence significative entre $p$ et $\pi_0$ \\
$\rightsquigarrow$ Il est crédible que $\pi\neq\pi_0$

\
\item Si $|t| < z_{\alpha} \; \Rightarrow \; $ Non-rejet de $H_0$ \\

$\rightsquigarrow$ Pas de différence significative entre $p$ et $\pi_0$ \\
$\rightsquigarrow$ On ne peut pas exclure que $\pi = \pi_0$

\end{itemize}
\end{itemize}
\end{textblock*}

\end{frame}

  %%%%%%%%%%%%%%%%%%%%%%%%%%%%%%%%%%%%%%%%%%%%%%%%%%%%%%%%%%%%%%%

\begin{frame}{Test de conformité de proportion}
\begin{textblock*}{\textwidth}(1cm,1.5cm)

\begin{center}{\bf \Large Exemple  } \end{center}
 
 
\begin{itemize}
\item Les conditions sont vérifiées :  $n\geq 30$, $n\pi_0\geq 5$ et $n(1-\pi_0)\geq 5$

\
\item $t=\frac{p-\pi_0}{\sqrt{\frac{\pi_0(1-\pi_0)}{n}}} \approx -2.24
$
\\
avec  $p=\frac{35}{500}$ proportion dans l'échantillon.

\
\item  $|t|>1.96$ on rejette  $H_0$

\
\item Comme $p<\pi_0$,  bouchons en liège 
du Portugal meilleurs

\
\item Degré de significativité 0.03
\end{itemize}

\end{textblock*}

\end{frame}
 
   %%%%%%%%%%%%%%%%%%%%%%%%%%%%%%%%%%%%%%%%%%%%%%%%%%%%%%%%%%%%%%%

\begin{frame}{Test d'homogénéité de proportions}
\begin{textblock*}{\textwidth}(1cm,2cm)

\begin{center}{\bf \Large Exemple  } \end{center}



 \
 
 Traitement contre bactérie du platane

\begin{itemize}
\item 126 platanes traités avec le produit A  : 99 guérisons
\item 130 platanes traités avec le produit B : 95 guérisons
\item Deux populations (traitement A / traitement B) indépendantes
\item Efficacité différente des deux traitements ?
\end{itemize}

\end{textblock*}

\end{frame}


   %%%%%%%%%%%%%%%%%%%%%%%%%%%%%%%%%%%%%%%%%%%%%%%%%%%%%%%%%%%%%%%

\begin{frame}{Test d'homogénéité de proportions}
\begin{textblock*}{\textwidth}(1cm,1.5cm)

\begin{center}{\bf \Large Notations} \end{center}



 \
 
 Traitement contre bactérie du platane

\begin{itemize}
\item $\pi_1$ et $\pi_2$ les taux de guérison dans ces deux populations 
\item  $\mathrm{P}_1$ et $\mathrm{P}_2$ les estimateurs sur des échantillons de taille $n_1$  et $n_2$
\item  estimation $p_1=99/126$ de $\pi_1$ et  estimation $p_2 =95/130$ de $\pi_2$

\
\item On teste \\ 
$H_0 : \pi_1=\pi_2$ contre $H_1 : \pi_1 \neq \pi_2$.
\end{itemize}

\end{textblock*}

\end{frame}
 
 %%%%%%%%%%%%%%%%%%%%%%%%%%%%%%%%%%%%%%%%%%%%%%%%%%%%%%%%%%%%%%%

\begin{frame}{Test d'homogénéité de proportions}
\begin{textblock*}{\textwidth}(1cm,1.5cm)

\begin{center}{\bf \Large Principe  } \end{center}



 \
On définit :  
\begin{itemize}
\item Estimateur de la proportion commune $\pi_1=\pi_2$ : 
$\mathrm{P}=\frac{n_1 \mathrm{P}_1 + n_2 \mathrm{P}_2}{n_1+n_2}$
\item Statistique de test :
$
T=\frac{\mathrm{P}_1-\mathrm{P}_2}{\sqrt{\mathrm{P}(1-\mathrm{P})} \sqrt{\frac{1}{n_1} + \frac{1}{n_2}} }$
\end{itemize}

Si
\begin{itemize}
\item $n_1\geq 30$, $n_2\geq 30$ 
\item  $n_1 \pi_1>5$,  $n_1(1-\pi_1)>5$
\item $n_2 \pi_2>5$,  $n_2(1-\pi_2)>5$
\end{itemize}

alors $T\sim \mathcal{N}(0;1)$ approximativement.

\end{textblock*}

\end{frame}

 %%%%%%%%%%%%%%%%%%%%%%%%%%%%%%%%%%%%%%%%%%%%%%%%%%%%%%%%%%%%%%%

\begin{frame}{Test d'homogénéité de proportions}
\begin{textblock*}{\textwidth}(1cm,1.5cm)

\begin{center}{\bf \Large Exemple  } \end{center}

 \
On calcule

\
\begin{itemize}
\item $\displaystyle p=\frac{99 + 95}{126 + 130}\approx 0.758$

\
\item $
\displaystyle t=\frac{p_1-p_2}{\sqrt{p(1-p)} \sqrt{\frac{1}{n_1} + \frac{1}{n_2}} } \approx 1.026
<1.96$
\end{itemize}

\

Pas de différence significative entre les efficacités
des deux produits

\

Conditions de validité vérifiées \emph{a porteriori} en remplaçant $\pi_1$ et $\pi_2$ par leur 
estimation commune $p$ : $\min(n_1, n_2) \min(p, 1-p) > 5$.


\end{textblock*}

\end{frame}

 %%%%%%%%%%%%%%%%%%%%%%%%%%%%%%%%%%%%%%%%%%%%%%%%%%%%%%%%%%%%%%%






%%%%%%%%%%%%%%%%%%%%%%%%%%%%%%%%%%%%%%%%%%%%%




\end{document}




 








