%\documentclass[12pt]{amsart}
%\usepackage{amsfonts,amssymb,epsfig}
\documentclass[12pt, a4paper]{article}

%%%%%%%%%% Packages %%%%%%%%%%

\usepackage{amsfonts,amssymb,amsmath}

\usepackage[utf8]{inputenc}

\usepackage[all]{xy}
\usepackage{delarray}

\usepackage[french]{babel}
\usepackage[T1]{fontenc}

\usepackage[colorlinks=true,allcolors=black]{hyperref}

%\pagestyle{empty}

%\marginparsep = -2cm
%\addtolength{\evensidemargin}{-2.2cm}
%\addtolength{\oddsidemargin}{-0.7cm}
%\setlength {\topmargin}{-1cm}
%\setlength {\textwidth}{16cm}
%\textheight 22cm

\textwidth=170mm
\textheight=254mm

\voffset=-18 mm
\hoffset=-15 mm     

\newcommand{\D}{\displaystyle}
\newcommand{\N}{{\mathbb N}}
\newcommand{\Z}{{\mathbb Z}}
\newcommand{\K}{{\mathbb K}}
\newcommand{\R}{{\mathbb R}}
\newcommand{\C}{{\mathbb C}}
%\pagenumbering{arabic}{-1}


%-----------------------------------------------------------------------------
\begin{document}


\noindent Université Paul Sabatier -- Toulouse III \hfill  Introduction aux Statistiques

\noindent Licence SdV - Parcours BCP

%noindent Introduction aux Statistiques

%\noindent Année 2005-2006
\bigskip
\bigskip

\begin{center}
{\huge {\bf Formulaire de cours}}
\end{center}

\bigskip
\noindent {\it Ce formulaire sera le seul document autorisé pour les évaluations de l'UE Introduction aux Statistiques. Il est permis de souligner ou de colorier mais pas de rajouter des commentaires.
Attention, ce document n'est pas un polycopié de cours mais un formulaire qui résume les notions vues en cours pour vous éviter d'apprendre des formules par coeur. }
%\thispagestyle{empty}
%\vspace{1cm}

\section{Variables aléatoires discrètes}

\subsection[Loi binomiale]{Loi binomiale $\mathcal{B}(n\,;\,p)$ avec $n\in\mathbb{N}$ et $p\in[0;1]$}

Une variable aléatoire discrète $X$ suit une loi binomiale de paramètres $n$ et $p$ si elle prend les 
valeurs  $\{0, 1, 2\hdots, n\}$ avec les probabilités 
$$
\Pr(X=k)=C^k_n p^k (1-p)^{n-k}\,, \quad \forall k\in\{0,1,\hdots , n\}\,.
$$
On rappelle que $C^k_n=\frac{n !}{k! (n-k)!}$ avec $k!=1\times 2\times 3\times\hdots\times k$ et $0!=1$. \\
Un calcul donne $E(X)=np$ et $Var(X)=np(1-p)$.


\subsection[Loi de Poisson]{Loi de Poisson $\mathcal{P}(\lambda)$ avec $\lambda>0$}

Une variable aléatoire discrète $X$ suit une loi de Poisson de paramètre $\lambda>0$ si l'ensemble de ses valeurs est
$\mathbb{N}$ avec les probabilités 
$$
\Pr(X=k)=\frac{\lambda^k}{k !} e^{-\lambda} \,, \quad \forall k\in\mathbb{N}.
$$
On a alors $E(X)=Var(X)=\lambda$.\\

\noindent {\bf Propriété :} Si $X$ suit une loi binomiale $\mathcal{B}(n,p)$ avec $n\geq 30$ ($n$ grand), 
$p\leq 0,1$ ($p$ petit) et $np\leq 10$ alors on peut supposer que $X$ suit approximativement une loi de Poisson $\mathcal{P}(\lambda)$ avec $\lambda=np$.

\section{Variables aléatoires continues}

\subsection[Loi uniforme sur un intervalle]{Loi Uniforme $U[a,b]$}
Une variable aléatoire continue $X$ suit une loi si sa densité de probabilité est la fonction
$f:\mathbb{R}\longrightarrow \mathbb{R}$ définie par 
$f(t)=\frac{1}{b-a}$ si $t \in [a,b]$ et $f(t)=0$ sinon. 
La fonction de répartition sera alors la fonction $F$ définie par $F(t)=0$ si $t<a$, $F(t)=\frac{t-a}{b-a}$
si $t\in [a,b]$ et $F(t)=1$ si $t>b$.
On a $E(X)=\frac{a+b}{2}$ et $Var(X)=\frac{(a-b)^2}{12}$.

\subsection[Loi exponentielle]{La loi exponentielle $\mathcal{E}(\lambda)$, avec $\lambda>0$}

Une variable aléatoire continue $X$ suit une loi exponentielle de paramètre $\lambda>0$ si sa densité de 
probabilité est la fonction $f:\mathbb{R}\longrightarrow \mathbb{R}$ donnée par  $f(t)=0$ si $t<0$ et
$f(t)=\lambda e^{-\lambda t}$  si $t\geq 0$.

La fonction de répartition sera alors la fonction $F$ définie par $F(t)=0$ si $t<0$ et 
$F(t)=1-e^{-\lambda t}$ si $t\geq 0$. Un petit calcul montre que $E(X)=\frac{1}{\lambda}$ et $Var(X)=\frac{1}{\lambda^2}$.\\
Les lois exponentielles vérifient la propriété $ \Pr(X>s+t|X>t) = \Pr(X>s)$ pour tout $s,t>0$. On dit qu'elles sont ``sans mémoire".


\subsection[Loi de Weibull]{La loi de Weibull $W(a,b)$ avec $a>0$ et $b>0$}

Une variable aléatoire continue $X$ suit une loi de Weibull de paramètres $a>0$ et $b>0$ si sa densité de probabilité est la 
fonction $f:\mathbb{R}\longrightarrow \mathbb{R}$ définie par  $f(t)=0$ si $t<0$ et
$f(t)=\frac{a}{b} \big( \frac{t}{b} \big)^{a-1} \mathrm{exp}\big( - \big( \frac{t}{b} \big)^{a} \big)$ si  $t\geq 0$.

La fonction de répartition sera alors la fonction $F$ définie par $F(t)=0$ si $t<0$ et 
$F(x)=1- \mathrm{exp}\big( - \big( \frac{t}{b} \big)^{a} \big)$ si $t\geq 0$. 
L'expression de la moyenne et de l'écart-type est un peu compliquée dans ce cas.
Notons que lorsque $a=1$, on retrouve la loi exponentielle.

\subsection[Loi normale]{La loi normale $\mathcal{N}(\mu,\sigma)$ avec $\mu\in \mathbb{R}$ et $\sigma>0$}

Une variable aléatoire continue $X$ suit une loi normale (ou loi de Gauss) $\mathcal{N}(\mu,\sigma)$ avec
$\mu\in \mathbb{R}$ et $\sigma>0$ si sa densité de probabilité est la 
fonction $f:\mathbb{R}\longrightarrow \mathbb{R}$ définie par  
$$
f(x)=\frac{1}{\sigma\sqrt{2\pi}} e^{-\frac{(x-\mu)^2}{2\sigma^2}}\,.
$$
On a alors  $E(X)=\mu$ et $Var(X)=\sigma^2$.

\noindent {\bf Propriété :} Si $X$ suit une loi $\mathcal{N}(\mu,\sigma)$ alors $Y=\frac{X-\mu}{\sigma}$ suit
une loi $\mathcal{N}(0,1)$.

\noindent {\bf Propriété :} Si $X_1$ suit une loi normale $\mathcal{N}(\mu_1,\sigma_1)$ et 
$X_2$ une loi normale $\mathcal{N}(\mu_2,\sigma_2)$ et si $X_1$ et $X_2$ sont indépendantes, alors
la variable $\alpha X_1 + \beta X_2$ (où $\alpha$ et $\beta$ sont deux réels quelconques) suit une loi
normale $\mathcal{N}(\alpha\mu_1+\beta\mu_2,\sqrt{\alpha^2\sigma_1^2+\beta^2\sigma_2^2})$.


\section{Estimation et tests statistiques sur moyennes }
On considère une population $\Omega$ et on note $\mathcal{E}_n$ l'ensemble de tous les échantillons de $n$
individus pris dans $\Omega$.

\subsection{Estimation de la moyenne et de l'écart-type} \label{EstimationMoyenne}
On prend une variable aléatoire $X : \Omega \longrightarrow \mathbb{R}$ et on note $\mu$ sa moyenne et
$\sigma$ son écart-type. 
Si on considère un échantillon de $n$ individus, on note $x=(x_1,\hdots,x_n)$ les valeurs de $X$ prises sur chaque individu de l'échantillon.
On définit alors $M_n : \mathcal{E}_n \longrightarrow \mathbb{R}$ 
et $S_n^2 : \mathcal{E}_n \longrightarrow \mathbb{R}$
par 
\begin{eqnarray*}
M_n(echantillon) & = & \bar{x}=\frac{x_1+\hdots x_n}{n} = \frac{\sum_{i=1}^n x_i}{n}\\
S_n^2(echantillon) & = & s_x^2=\frac{1}{n-1}\sum_{i=1}^n (x_i-\bar{x})^2 = \frac{n}{n-1}\big( \frac{\sum_{i=1}^n x_i^2}{n} - \bar{x}^2 \big)
=\frac{\sum_{i=1}^n x_i^2}{n-1} -  \frac{n}{n-1}\bar{x}^2
\end{eqnarray*}

Les variables aléatoires $M_n$ et $S_n^2$ sont les estimateurs respectivement de $\mu$ et $\sigma^2$. 

Les valeurs $\bar{x}$ et $s_x^2$ sont les estimations de $\mu$ et $\sigma^2$ à partir de l'échantillon choisi.\\

%On peut montrer que 
%$\displaystyle s_x^2=\frac{n}{n-1}\big( \frac{\sum_{i=1}^n x_i^2}{n} - \bar{x}^2 \big)=\frac{\sum_{i=1}^n x_i^2}{n-1} -  \frac{n}{n-1}\bar{x}^2$.\\

\noindent {\bf Propriété :} On a $E(M_n)=\mu$ et $Var(M_n)=\sigma^2/n$.
Si $n$ est grand, $M_n$ suit approximativement une loi normale $\mathcal{N}(\mu;\sigma/\sqrt{n})$. \\

\noindent {\bf Propriété :} On a $E(S_n^2)=\sigma^2$ et %$Var(S_n^2)=\frac{2\sigma^4}{n-1}$.
la variable $\frac{(n-1)S_n^2}{\sigma^2}$ suit une loi du $\chi^2$ à $n-1$ degrés de liberté.\\

%\subsection{Statistique de test}

On considère maintenant la variable aléatoire $\displaystyle T=\frac{M_n-\mu}{S_n/\sqrt{n}}$ 

\noindent {\bf Propriété :} Si $n$ est grand ($n\geq 30$ en général) alors $T$ suit approximativement une
loi normale centrée réduite $\mathcal{N}(0;1)$.\\

\noindent {\bf Propriété :} Si $n$ est petit ($n < 30$), mais si $X$ est distribuée selon une loi normale alors $T$ suit une loi de Student
de degré de liberté égal à $n-1$.

\subsection{Test de conformité sur les moyennes}
On prend une variable aléatoire $X : \Omega \longrightarrow \mathbb{R}$ et on note $\mu$ sa moyenne et
$\sigma$ son écart-type ; ces deux paramètres étant inconnus, à estimer sur l'échantillon. %On utilise les notations et propriétés du paragraphe \ref{EstimationMoyenne}.\\
 
\subsubsection*{\it Hypothèses}
\begin{center}
$
\begin{array}{ll}
& H_0: \mu = \mu_0 \mbox{ (où } \mu_0 \mbox{ est la valeur de référence donnée)} \\
\mbox{vs } & H_1: \mu \neq \mu_0  \mbox{ (ou } H_1: \mu > \mu_0 \mbox{ ou } H_1: \mu < \mu_0)
\end{array}
$
\end{center}

\noindent {\it Statistique de test } \hspace{3cm} $\displaystyle T=\frac{M_n-\mu_0}{S_n/\sqrt{n}}$.

\subsection{Test d'homogénéité sur les moyennes}
On prend deux variables aléatoires $X_1$ et $X_2$, on note $\mu_1$ et $\mu_2$ les moyennes, $\sigma_1$ et
$\sigma_2$ les écart-types.
On note $M_1$ et $S_1$ les estimateurs de $\mu_1$ et $\sigma_1$ sur des échantillons de taille $n_1$.
De même, on note $M_2$ et $S_2$ les estimateurs de $\mu_2$ et $\sigma_2$ sur des échantillons de taille $n_2$.

\subsubsection*{\it Hypothèses}
\begin{center}
$\begin{array}{ll}
 & H_0: \mu_1 = \mu_2 \\
\mbox{vs }& H_1: \mu_1 \neq \mu_2  \mbox{ (ou } H_1 : \mu_1 > \mu_2 \mbox{ ou } H_1 : \mu_1 < \mu_2 \mbox{)} 
\end{array}$
\end{center}

%\noindent {\it Statistique de test :} grands ou petits échantillons ?

\subsubsection*{\it Statistique de test sur grands échantillons}
Si $n_1$ et $n_2$ sont grands (supérieurs à 30),
$\displaystyle T=\frac{M_1-M_2}{\sqrt{\frac{S_1^2}{n_1}+\frac{S_2^2}{n_2}} }$ suit une loi normale $\mathcal{N}(0;1)$.

\subsubsection*{\it Statistique de test sur petits échantillons} 
Si $n_1$ et/ou $n_2$ sont petits (inférieurs à 30),
on note alors $\displaystyle S^2=\frac{(n_1-1)S_1^2+(n_2-1)S_2^2}{n_1 + n_2-2}$ et 
$\displaystyle T=\frac{M_1-M_2}{\sqrt{\frac{S^2}{n_1}+\frac{S^2}{n_2}} }$.
Si $X_1$ et $X_2$ sont distribuées selon une loi normale et si $\sigma_1=\sigma_2$ alors $T$ suit
une loi de Student à $n_1+n_2-2$ degrés de liberté.

\section{Estimation et tests statistiques sur proportions}

\subsection{Estimation de proportions}\label{EstimationProportion}

On note $\pi$ la proportion d'individus de $\Omega$ qui ont une certaine caractéristique $\mathcal{C}$.

On construit alors l'estimateur de la proportion $\pi$, $\mathrm{P}_n : \mathcal{E}_n \longrightarrow \mathbb{R}$ par

$\mathrm{P}_n$ (échantillon) $=p = card(\mathcal{C}) / card(\Omega)$ : proportion d'individus de l'échantillon qui présentent la 
caractéristique $\mathcal{C} $.\\

\noindent {\bf Propriété :} On a  $E(\mathrm{P}_n)=\pi$ et 
$Var(\mathrm{P}_n)=\frac{\pi(1-\pi)}{n}$.\\

\noindent {\bf Propriété  : } Si $n$ est grand ($n\geq 30$) et si $n\pi\geq 5$ et $n(1-\pi)\geq 5$ alors on peut 
supposer
que $\mathrm{P}_n$ suit approximativement une loi normale $\mathcal{N}(\pi;\sqrt{\pi(1-\pi)/n})$.\\

\noindent {\bf Propriété  : } Si $n$ est grand ($n\geq 30$) et si $n\pi\geq 5$ et $n(1-\pi)\geq 5$ alors la variable aléatoire
$$
T=\frac{\mathrm{P}_n - \pi}{\sqrt{\frac{\mathrm{P}_n (1-\mathrm{P}_n)}{n}}}
$$ 
suit approximativement une loi normale centrée réduite $\mathcal{N}(0;1)$.

\subsection{Test de conformité sur les proportions}

%On utilise les mêmes notations et définitions qu'au paragraphe \ref{EstimationProportion}.\\

%\noindent {\it Hypothèse :} $\pi = \pi_0$ où $\pi_0$ est donnée. \\

\subsubsection*{\it Hypothèses}
\begin{center}
$
\begin{array}{ll}
& H_0: \pi = \pi_0 \mbox{ (où } \pi_0 \mbox{ est la valeur de référence donnée)} \\
\mbox{vs } & H_1: \pi \neq \pi_0  \mbox{ (ou } H_1: \pi > \pi_0 \mbox{ ou } H_1: \pi < \pi_0)
\end{array}
$
\end{center}

\noindent {\it Statistique de test } \hspace{3cm} $\displaystyle T=\frac{ P_n-\pi_0}{\sqrt{\frac{\pi_0(1-\pi_0)}{n}}}$. \\
Si $n\geq 30$, $n\pi_0>5$
et  $n(1-\pi_0)>5$ alors $T$ suit approximativement une loi normale $\mathcal{N}(0;1)$.

\subsection{Test d'homogénéité sur les proportions}

On note $\pi_1$ la proportion d'individus d'une population $\mathcal{P}_1$ qui présentent une certaine 
caractéristique et $\mathrm{P}_1$ son estimateur sur des échantillons de taille $n_1$.
De même on note $\pi_2$ la proportion d'individus d'une population $\mathcal{P}_2$ qui présentent une certaine caractéristique et $\mathrm{P}_2$ son estimateur sur des échantillons de taille $n_2$.

%\noindent {\it Hypothèse :} $\pi_1 = \pi_2$.\\

\subsubsection*{\it Hypothèses}
\begin{center}
$\begin{array}{ll}
 & H_0: \pi_1 = \pi_2 \\
\mbox{vs }& H_1: \pi_1 \neq \pi_2  \mbox{ (ou } H_1 : \pi_1 > \pi_2 \mbox{ ou } H_1 : \pi_1 < \pi_2 \mbox{)} 
\end{array}$
\end{center}

\noindent {\it Statistique  de test :} 
On note $\displaystyle \mathrm{P}=\frac{n_1 \mathrm{P}_1 + n_2 \mathrm{P}_2}{n_1+n_2}$
et $\displaystyle T=\frac{\mathrm{P}_1-\mathrm{P}_2}{\sqrt{\mathrm{P}(1-\mathrm{P})} \sqrt{\frac{1}{n_1} + \frac{1}{n_2}} }$. \\
Si $n_1\geq 30$, $n_2\geq 30$, $n_1 \pi_1>5$,  $n_1(1-\pi_1)>5$, $n_2 \pi_2>5$ et  $n_2(1-\pi_2)>5$
alors la variable $T$ suit approximativement une loi normale $\mathcal{N}(0;1)$.

\section{Tests stastistiques sur répartitions}

\noindent {\it Statistique de test du $\chi^2$ :} 
$$
Q=\sum\frac{ ({\mbox { effectif observé }} -  {\mbox { effectif théorique }})^2}
 { {\mbox { effectif théorique }}} \,
$$
Les effectifs théoriques sont calculés selon la problématique. 

\subsection[Test du chi-2 de conformité]{Test du $\chi^2$ de conformité}
On considère une population dans laquelle chaque individu peut être classé dans une catégorie parmi $k$. 
On note $C_1$,....., $C_k$ ces catégories supposées disjointes et $\pi_1$, $\pi_2$, ....,$\pi_k$ les proportions des $k$ catégories dans la population (ces proportions sont inconnues). \\

\noindent {\it Hypothèses :} 
\begin{itemize}
\item[] $H_0 : \pi_1=p_1, \;\hdots,\; \pi_k=p_k$ où $p_1, \hdots, p_k$ sont
des valeurs de référence données,
\item[] vs $H_1 : \pi_1\neq p_1 \mbox{ ou } \;\hdots,\;  \mbox{ ou } \pi_k \neq p_k$. \\
\end{itemize}

%\noindent {\it Statistique  de test :} 
%
%Q=\sum\frac{ ({\mbox { effectif observé }} -  {\mbox { effectif théorique }})^2}
% { {\mbox { effectif théorique }}} \,.
%$$

Sur un échantillon de taille $n$, on comptabilise $n_1$ individus de la $1^{\text{ère}}$ catégorie, 
$n_2$ de la $2^e$ catégorie, .... , $n_k$ de la $k^{\text{ème}}$ catégorie ; ce sont les effectifs observés. On calcule les effectifs théoriques : $np_1$, ....., $np_k$.

La valeur de $Q$ sur cet échantillon est alors 
$$
q=\sum_{i=1}^k \frac{(n_i- p_i n)^2}{p_i n}
$$

Si $n\geq 30$ et $np_i > 5$ pour tout $i$ alors la variable $Q$ suit une loi du $\chi^2$ à $(k-1)$ d.d.l.

\subsection[Test du chi-2 d'homogénéité]{Test du $\chi^2$ d'homogénéité}

On considère deux populations $\mathcal{P}$ et $\mathcal{P}'$ dans lesquelles chaque individu peut être classé dans une catégorie parmi $k$. 
On note $C_1$,....., $C_k$ et $\pi_1$, $\pi_2$, ....,$\pi_k$ les catégories et les proportions de $\mathcal{P}$. De même pour  $C_1'$,....., $C_k'$ et $\pi_1'$, $\pi_2'$, ....,$\pi_k'$.\\

\noindent {\it Hypothèses :}  
\begin{itemize}
\item[] $H_0 : \pi_1=\pi'_1, \;\hdots,\; \pi_k=\pi'_k$
\item[] vs  $H_1 : \pi_1 \neq \pi'_1  \mbox{ ou } \;\hdots,\; \mbox{ ou } \pi_k \neq \pi'_k$. \\
\end{itemize}
%\noindent {\it Statistique de test :} 
%$$
%Q=\sum\frac{ ({\mbox { effectif observé }} -  {\mbox { effectif théorique }})^2}
% { {\mbox { effectif théorique }}} \,.
%$$

On dispose de deux échantillons indépendants, de tailles $n$ et $n'$, et
on observe la répartition des effectifs suivant les $k$ catégories : $n_1$, $n_2$, .... , $n_k$ et  $n'_1$, $n'_2$, .... , $n'_k$.

Les proportions théoriques $p_1, \hdots, p_k$ sont calculées par 
$\displaystyle p_i = \frac{n_i+n'_i}{n+n'}$.
Les effectifs théoriques sont alors $np_1$, $np_2$, ... , $np_k$ pour le premier échantillon et $n'p_1$, $n'p_2$, ... , $n'p_k$ pour le second échantillon.

La valeur de $Q$ sur ces deux échantillons est alors
$$
q=\sum_{i=1}^k \frac{(n_i- p_i n)^2}{p_i n}  + \sum_{i=1}^k \frac{(n'_i- p_i n')^2}{p_i n'}\,.
$$

Si $n\geq 30$, $n' \geq 30$, $np_i > 5$ et $n'p_i > 5$ pour tout $i$ 	alors la variable $Q$ suit une 
loi du $\chi^2$ à $k-1$ degrés de liberté.

\subsection[Test du chi-2 d'indépendance]{Test du $\chi^2$ d'indépendance}

On considère une population $\mathcal{P}$ dans laquelle chaque individu est étudié selon 2 caractéristiques : il est classé dans une première catégorie parmi $L$ (pour la première caractéristique) et une deuxième catégorie parmi $C$ (pour la deuxième caractéristique). \\
  
\noindent {\it Hypothèses :} 
\begin{itemize} 
\item[] $H_0$ : les deux caractéristiques sont indépendantes 
\item[] vs $H_1$ : les deux caractéristiques ne sont pas indépendantes. \\
\end{itemize}

\noindent On note : 
\begin{itemize}
\item[] $n_{ij}$ le nombre d'individus dans les catégories $i$ et $j$ ($i=1,..., L$ ; $j=1,..., C$), 
\item[] $n_{i.}$ le nombre d'individus dans la catégorie $i$ : $n_{i.} = \sum_{j=1}^C n_{ij}$,
\item[] $n_{.j}$ le nombre d'individus dans la catégorie $j$ : $n_{.j} = \sum_{i=1}^L n_{ij}$,
\item[] $n$ le nombre total d'individus dans l'échantillon : $n = \sum_{i=1}^L \sum_{j=1}^C n_{ij}$.
\end{itemize}
Les effectifs théoriques sous l'hypothèse d'indépendance sont calculés par 
$\displaystyle e_{ij} = \frac{n_{i.} \times n_{.j}}{n}$ \\

La valeur de $Q$ sur l'échantillon est alors : \;\;
$
\displaystyle q=\sum_{i=1}^L \sum_{j=1}^C \frac{(n_{ij}- e_{ij})^2}{e_{ij}}
$ \\

Si $n\geq 30$ et $e_{ij} > 5$ pour tout $(i,j)$ alors $Q$ suit une 
loi du $\chi^2$ à $(L-1)(C-1)$ d.d.l.

\section{Annexes : autres variables aléatoires continues}

\subsection[Loi log-normale]{La loi log-normale $Log-\mathcal{N}(\mu,\sigma)$ avec $\mu\in \mathbb{R}$ et $\sigma>0$}

Une variable aléatoire continue $X$ suit une loi Log-Normale (ou loi de Galton) 
$Log-\mathcal{N}(\mu,\sigma)$ avec $\mu\in \mathbb{R}$ et $\sigma>0$ si $\ln(X)$ suit une loi normale $\mathcal{N}(\mu,\sigma)$.
La densité de probabilité de $X$ est donc la fonction 
$f : ]0\,;\,+\infty[ \longrightarrow \mathbb{R}$ définie par  
$f(x)=\frac{1}{\sigma x\sqrt{2\pi}} e^{-\frac{(\ln (x) -\mu)^2}{2\sigma^2}}$.

Un calcul donne $E(X)=e^{(\mu+\sigma^2/2)}$,  $Var(X)=(e^{\sigma^2}-1) e^{2\mu+\sigma^2}$, et le mode vaut $e^{(\mu-\sigma^2)}$.

\subsection[Loi du chi-2]{La loi du $\chi^2$}

On dit qu'une v.a. continue $Y$ suit une loi du $\chi^2$ à $d$ degrés de liberté si on peut écrire
$$
Y=X_1^2 + X_2^2 + \hdots + X_d^2
$$
où les $X_i$ sont des variables aléatoires indépendantes qui suivent une loi normale centrée réduite $\mathcal{N}(0;1)$. On peut démontrer facilement que $E(Y)=d$ et $Var(Y)=2d$.

\subsection{La loi de Student}
On dit que la variable aléatoire continue $T$ suit une loi de Student à $d$ degrés de liberté si on peut écrire $ T=\frac{X}{\sqrt{Y/d}}$ 
où la variable aléatoire $X$ suit une loi normale $\mathcal{N}(0;1)$, 
la variable aléatoire $Y$ suit une loi du $\chi^2$ à $d$ degrés de liberté, 
$X$ et $Y$ sont indépendantes.

On a alors $E(T)=0$ si $d>1$ et $Var(T)=\frac{d}{d-2}$ si $d>2$.

On peut démontrer que si $T$ suit une loi de Student à degré de liberté $d$ grand ($d>30$ par exemple) 
alors $T$ suit approximativement une loi normale centrée réduite $\mathcal{N}(0;1)$.

\end{document}
