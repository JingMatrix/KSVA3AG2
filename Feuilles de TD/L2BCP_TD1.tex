\documentclass[11pt]{article}
\usepackage{amsfonts,amssymb,epsfig}
%\usepackage[latin1]{inputenc}
\usepackage[all]{xy}
\usepackage{amsmath}
\usepackage{delarray}
\usepackage{pdfpages}
\usepackage[frenchb]{babel}

\usepackage[utf8]{inputenc} 

\newtheorem{thm}{Theorem}[section]
\newtheorem{cor}[thm]{Corollary}
\newtheorem{lem}[thm]{Lemma}
\newtheorem{prop}[thm]{Proposition}
%\theoremstyle{definition}
\newtheorem{defn}[thm]{Definition}
%\theoremstyle{remark}
\newtheorem{rem}[thm]{Remark}
\numberwithin{equation}{section}
\newtheorem{example}[thm]{Example}

\newcommand{\D}{\displaystyle}
\newcommand{\N}{{\mathbb N}}
\newcommand{\Z}{{\mathbb Z}}
\newcommand{\K}{{\mathbb K}}
\newcommand{\R}{{\mathbb R}}
\newcommand{\C}{{\mathbb C}}

\setlength{\textwidth}{17.2cm}
\setlength{\textheight}{24cm}
\hoffset=-2.2cm
\voffset=-2.5cm

%-----------------------------------------------------------------------------
\begin{document}
\noindent Université Paul Sabatier -- Toulouse III
\hfill L2 BCP\\
\noindent Introduction aux Statistiques
%\hfill Année 2022-2023

\bigskip
\bigskip


\begin{center}
{\bf \large TD 1 : Variables aléatoires continues}\\
{\bf \large }
\end{center}

\bigskip
%\bigskip

\subsection*{Exercice 1 :}
Dans le cadre d'une expérience, des scientifiques ont besoin de disposer de nombreuses larves de 
diptères. Le poids de ces larves suit approximativement une loi normale de moyenne 3.4g d'écart type 
de 0.42. On note $X$ la variable aléatoire représentant le poids en grammes d'une larve prise au hasard.

\begin{enumerate}
\item Quelle est la probabilité qu'une larve prise au hasard pèse 3 grammes ?
\item Seules les larves d'un poids supérieur à la moyenne sont éventuellement conservées pour l'élevage de la génération suivante. Quelle est la probabilité qu'une larve prise au hasard puisse être utilisée pour la reproduction ?
\item Calculer la probabilité qu'une larve pèse moins de 3.75 grammes.
\item Pour l'expérience considérée, les scientifiques ont besoin de larves de plus de 3g. Quelle proportion de larves pourront-ils utiliser ?
\item On note $Z$ la variable aléatoire représentant le poids total de 50 larves prises au hasard de 
façon indépendante. Quelle est la loi de probabilité de $Z$ ?
\end{enumerate}

\bigskip
\subsection*{Exercice 2 :} 
Les moustiques sont attirés par le dioxyde de carbone que nous dégageons lorsque nous respirons.
On note $Y$ la variable aléatoire qui représente la quantité (en grammes) de CO2 dégagée par un être 
humain dans des conditions normales, en une heure. 
On sait que $Y$ suit une loi normale $\mathcal{N}(22;5)$ de moyenne $\mu=22 \,g$ et d'écart-type 
$\sigma=5 \,g$. 
\begin{enumerate}
\item Donnerr la quantité médiane de CO2 dégagée par un être humain en une heure.
\item Expliquer pourquoi $Pr(Y\leq 17) = Pr(Y\geq 27)$. 
\item Calculer la probabilité qu'un individu dégage une quantité de CO2 comprise entre
$20\,g$ et  $26\,g$.
\item Certaines personnes dégagent naturellement beaucoup de CO2 et attirent donc plus les moustiques. Déterminer le seuil $M$ tel que seulement $5\%$ des êtres humains dégagent une quantité de CO2 par 
heure supérieure à $M$.
\item Déterminer un intervalle de type $[\mu - a ; \mu + a]$ tel que $95\,\%$ des individus 
dégagent une quantité de CO2 comprise entre $\mu - a$ et $\mu + a$.  \\
\end{enumerate}



\bigskip
\subsection*{Exercice 3 :} 

L'{\it analyse de donn\'ees de survie} est l'\'etude du temps \'ecoul\'e avant l'apparition d'un 
ph\'enom\`ene en g\'en\'eral important (mort, accident, mutation, gu\'erison, panne d'un appareil \'electronique, etc.).

La plupart des tumeurs canc\'ereuses sont la cons\'equence d'une mutation du g\'enome cellulaire. 
Dans cet exercice, on s'int\'eresse au temps pass\'e avant l'apparition de la mutation des cellules ; 
c'est un ph\'enom\`ene al\'eatoire. 

On note $T$ la variable al\'eatoire repr\'esentant le {\it temps de vie} d'une cellule jusqu'\`a 
l'apparition d'une mutation canc\'ereuse. 
La fonction $S : \mathbb{R}\longrightarrow \mathbb{R}$ d\'efinie par $S(t)=Pr(T>t)$ s'appelle la 
{\it fonction de survie}.

On suppose que la variable $T$ suit une loi exponentielle $\mathcal{E}(\lambda)$ de param\`etre 
$\lambda=10^{-2}$. 
\begin{enumerate}
\item Donner l'expression de la fonction $S(t)$. Donner le valeurs des param\`etres 
$E(T)$ et $\sigma(T)$.
\item Calculez $Pr(T<40)$. 
\item D\'eterminez la m\'ediane de $T$.
\item On consid\`ere un r\'eel $\varepsilon>0$. 
Calculer la probabilit\'e conditionnelle $Pr(T\geq t + \varepsilon\, | \, T>t)$.
Comparer cette probabilité avec $Pr(T>\varepsilon)$. 
Est-ce que modéliser la loi de probabilité de $T$ par une loi exponentielle paraît pertinent ? 
\item Il faut modéliser la loi de probabilité de $T$ par une loi plus générale que la loi 
exponentielle. On tente une loi de Weibull $W(a,b)$ de paramètres $a>0$ et $b>0$. Pensez-vous que
$a>1$ ou $a<1$ ? 
\end{enumerate}

\iffalse %%%%%%%%%%%%%%%%%%%%%%%%%%%%%%%
{\bf 2)} On suppose maintenant que $T$ suit une loi de Weibull $W(a,b)$. Donnez $S(t)$.

On fixe un r\'eel $\varepsilon>0$. Donnez l'expression de la probabilit\'e $Pr(T\geq t + \varepsilon\, | \, T>t)$.
En supposant que $\frac{\varepsilon}{t}\approx 0$, on pourrait alors d\'emontrer (mais on ne le fera pas),
que  
$Pr(T\geq t + \varepsilon\, | \, T>t)\approx exp(-\frac{a\varepsilon}{b^a} t^{a-1})$. Etudiez l'effet du 
vieillissement \`a partir de cette probabilit\'e.

{\bf 3)} Un organe est compos\'e de $N$ cellules. On note $T_1,\hdots, T_N$ les variables al\'eatoires repr\'esentant les
temps de vie des cellules jusqu'\`a l'apparition d'une mutation. 
On suppose que ces $N$ variables sont ind\'ependantes et sont toutes distribu\'ees selon la loi exponentielle de la question 1).
On suppose que la premi\`ere mutation d'une des cellules suffit pour d\'eclencher une tumeur. On note $X$ la dur\'ee de vie de l'organe. 
On a alors $X=\min \{T_1,\hdots, T_N\}$.
D\'eterminez la fonction de survie $S_X(t)$ de l'organe. Calculez $S_X(40)$ pour $N=100$.\\

\fi %%%%%%%%%%%%%%%%%%%%%%%%%%







\end{document}
