\documentclass{article}
\usepackage{amsfonts,amssymb,epsfig}
%\usepackage[latin1]{inputenc}
\usepackage[all]{xy}
\usepackage{amsmath}
\usepackage{delarray}
\usepackage{pdfpages}
\usepackage[frenchb]{babel}

\usepackage[utf8]{inputenc} 

\newtheorem{thm}{Theorem}[section]
\newtheorem{cor}[thm]{Corollary}
\newtheorem{lem}[thm]{Lemma}
\newtheorem{prop}[thm]{Proposition}
%\theoremstyle{definition}
\newtheorem{defn}[thm]{Definition}
%\theoremstyle{remark}
\newtheorem{rem}[thm]{Remark}
\numberwithin{equation}{section}
\newtheorem{example}[thm]{Example}

\newcommand{\D}{\displaystyle}
\newcommand{\N}{{\mathbb N}}
\newcommand{\Z}{{\mathbb Z}}
\newcommand{\K}{{\mathbb K}}
\newcommand{\R}{{\mathbb R}}
\newcommand{\C}{{\mathbb C}}

\setlength{\textwidth}{17.2cm}
\setlength{\textheight}{24cm}
\hoffset=-2.2cm
\voffset=-2.5cm

%-----------------------------------------------------------------------------
\begin{document}
\noindent Université Paul Sabatier -- Toulouse III
\hfill L2 BCP\\
\noindent Introduction aux Statistiques
%\hfill Année 2022-2023

\bigskip
\bigskip


\begin{center}
{\bf \large TD 2 : Estimations et tests de comparaison de moyennes}\\
{\bf \large }
\end{center}

\bigskip
%\bigskip



\subsection*{  Exercice 1 :  (Examen Janvier 2016)} 

\noindent La bosse des dromadaires constitue une  réserve énergétique importante, pleine de matières grasses (acides
palmitiques, acide oléitique...). 

\noindent On note $X$ la variable aléatoire qui représente le taux d'acide palmitique dans la bosse d'un dromadaire pris au hasard. On note également $\mu$ et $\sigma$ la moyenne et l'écart-type de $X$.
Sur un échantillon de 120 dromadaires, on a mesuré le taux d'acide palmitique présent dans la bosse de chaque
dromadaire. On a ainsi obtenu la série statistique $x_1,\hdots, x_{120}$. On a alors calculé 
$\displaystyle \sum_{k=1}^{120} x_k \approx 36$ et $\displaystyle \sum_{k=1}^{120} x_k^2 \approx 12$.
\begin{enumerate}
\item Donnez une estimation de $\mu$ et $\sigma$.
\item Donnez un intervalle de confiance à $95\%$ (puis à $98\%$) de $\mu$.
\item On suppose que $X$ suit une loi normale $\mathcal{N}(\mu ; \sigma)$. Reprenez la question précédente en supposant que l'échantillon contient 17 dromadaires avec $\displaystyle \sum_{k=1}^{17} x_k=5.4$  et $\displaystyle \sum_{k=1}^{17} x_k^2=2.23$.
%\item Question subsidiaire pour le TD : On suppose que $X$ suit une loi normale $\mathcal{N}(\mu ; \sigma)$. On prendra pour valeurs de $\mu$ et de $\sigma$ les estimations calculées à la question 1. 
%\begin{enumerate}
%\item Donnez la médiane de $X$. 
%\item Calculez la probabilité que la bosse d'un dromadaire pris au hasard ait un taux d'acide palmitique supérieur à 0.41. \\
%\end{enumerate}
\end{enumerate}



\bigskip




\subsection*{ Exercice 2 : } 

\noindent On sait que le taux d'hémoglobine moyen chez l'homme est $\mu_0= 15\, g/dL$ de sang.

\begin{enumerate}
\item Un scientifique qui a de l'intuition décide d'étudier l'hémoglobine chez les coureurs cyclistes professionnels. 
On note $\mu$ le taux d'hémoglobine moyen chez les cyclistes et $\sigma$ l'écart-type. Il a alors mesuré le taux d'hémoglobine de chaque individu d'un échantillon de $50$ cyclistes et a calculé la moyenne $\bar{x}\approx 15.8$ et l'écart-type empirique $s_x\approx 2.35$. \\
Utilisez un test statistique avec le risque d'erreur $\alpha=0.05$ pour tester l'hypothèse $(H_0) \,\mu=\mu_0$ contre l'hypothèse alternative $(H_1)\, \mu\neq\mu_0$. Peut-on conclure que le taux d'hémoglobine moyen des cyclistes est anormal ? Donnez le degré de signification si nécessaire.

\item On note $\alpha$ le risque d'erreur (ou {\it risque de première espèce}) ;  c'est la probabilité de
rejeter l'hypothèse alors qu'elle est correcte. On note $\beta$ le risque de {\it deuxième espèce} ; c'est la probabilité de ne pas rejeter l'hypothèse $(H_0)$ alors qu'elle est fausse. La {\it puissance} du test est alors la probabilité de rejeter $(H_0)$ alors qu'elle est effectivement fausse. \\
Représentez géométriquement ces trois probabilités. \\
Si $\alpha$ est pris très petit, que devient $\beta$ ?

\item Un chercheur invente un produit à base d'une molécule connue pour favoriser l'augmentation 
du taux d'hémoglobine. Il cherche à savoir si son produit est vraiment efficace.
On note $\tilde{\mu}$ le taux
d'hémoglobine moyen lorsqu'on consomme ce produit, et $\tilde{\sigma}$ l'écart-type. \\
Sur un échantillon de $40$ individus auxquels le chercheur a injecté son produit, il a  calculé la moyenne 
$\bar{y}\approx 15.6$ et l'écart-type empirique $s_y\approx 1.5$. \\
Testez l'hypothèse $(H_0) \,\tilde{\mu}=\mu_0$ contre l'hypothèse alternative $(H_1)\, \tilde{\mu} > \mu_0$ pour vérifier si ce produit est efficace. Expliquez le choix de  cette hypothèse alternative. \\
Expliquez pourquoi ce test est "plus puissant" que celui fait au 1).

\item Refaites le test du 1) en supposant cette fois que l'échantillon est formé
de 20 cyclistes. Quelle hypothèse supplémentaire doit-on avoir pour que ce soit correct ?\\
\end{enumerate}

\bigskip


\subsection*{ Exercice 3 : (Examen Juin 2022)} 
On se propose d'étudier l'effet de l'âge sur le temps de réaction (en millisecondes) entre un 
stimulus visuel et une  réponse motrice.

On a mesuré le temps de réaction de $52$ individus de $30$ ans. On a obtenu un temps de réaction moyen 
de $m_1=255$ et un écart-type empirique de $s_1=32$.

On a mesuré le temps de réaction de $35$ individus de $55$ ans. 
On a obtenu un temps moyen de réaction de $m_2=264$ et un écart-type empirique de $s_2=35.5$.  \\

Cette expérience permet-elle de conclure à une différence de temps de réaction entre les âges 
$30$ et $50$ ans ? Donner le degré de signification si nécessaire.

\bigskip


\subsection*{ Exercice 4 : } 
Le chat est un gros dormeur ! On s'intéresse à la durée quotidienne de sommeil des chats et on voudrait savoir si les chats de campagne et de ville dorment autant. 
On a relevé, pour un jour donné, la durée de sommeil de $15$ chats de campagne et de $16$ chats de ville.  

Sur l'échantillon des chats de campagne, on a obtenu une durée moyenne $\overline{x} \approx 16.44$ et un écart-type $s_x \approx 2.9$. 

Sur l'échantillon des chats de ville, on a obtenu les résultats intermédiaires suivants (où $y_i$ représente la durée de sommeil d'un 
chat de ville $i$, avec $i=\{1, ...,16\}$) : 
$$
\sum_{i=1}^{16} y_i = 310 \;\;\;\;\;\; \sum_{i=1}^{16} y_i^2 = 6128 
$$
%Sur l'échantillon des chats de ville, on a obtenu une durée moyenne $\overline{y} \approx 19.21$ et un écart-type $s_y \approx 2.85$. \\
\begin{enumerate}
\item Calculer la moyenne et l'écart-type de la durée de sommeil pour l'échantillon des chats de ville.
\item A l'aide d'un test statistique (au risque $5\%$), peut-on conclure que les chats de ville et 
campagne dorment autant ? \\
Quelles sont les conditions pour que ce test soit applicable ? \\
Donner le degré de signification s'il y a lieu.\\
\end{enumerate}










\end{document}
