\documentclass{article}
\usepackage{amsfonts,amssymb,epsfig}
%\usepackage[latin1]{inputenc}
\usepackage[all]{xy}
\usepackage{amsmath}
\usepackage{delarray}
\usepackage{pdfpages}
\usepackage[frenchb]{babel}

\usepackage[utf8]{inputenc} 

\newtheorem{thm}{Theorem}[section]
\newtheorem{cor}[thm]{Corollary}
\newtheorem{lem}[thm]{Lemma}
\newtheorem{prop}[thm]{Proposition}
%\theoremstyle{definition}
\newtheorem{defn}[thm]{Definition}
%\theoremstyle{remark}
\newtheorem{rem}[thm]{Remark}
\numberwithin{equation}{section}
\newtheorem{example}[thm]{Example}

\newcommand{\D}{\displaystyle}
\newcommand{\N}{{\mathbb N}}
\newcommand{\Z}{{\mathbb Z}}
\newcommand{\K}{{\mathbb K}}
\newcommand{\R}{{\mathbb R}}
\newcommand{\C}{{\mathbb C}}

\setlength{\textwidth}{17.2cm}
\setlength{\textheight}{24cm}
\hoffset=-2.2cm
\voffset=-2.5cm

%-----------------------------------------------------------------------------
\begin{document}
\noindent Université Paul Sabatier -- Toulouse III
\hfill L2 BCP\\
\noindent Introduction aux Statistiques
%\hfill Année 2022-2023

\bigskip
\bigskip


\begin{center}
{\bf \large TD 3 : Estimations et tests de comparaison de proportions}\\
{\bf \large }
\end{center}

\bigskip
\bigskip


\subsection*{ Exercice 1 : } 
De nombreux sondages sont réalisés en politique au moment des élections. Par exemple, pour les élections américaines de 2016, le dernier sondage sur les intentions de vote a donné les résultats suivants : sur $1055$ personnes interrogées, $506$ déclarent qu'ils voteront pour Clinton, et 454 pour Trump. \\
Estimez la proportion d'intentions de votes pour chaque candidat, et donnez un intervalle de confiance à $95 \%$ correspondant.\\
Au vu des résultats de ce sondage, les journalistes donnent Clinton gagnante. Qu'en pensez-vous ?  \\

\bigskip 

\subsection*{ Exercice 2 : } 
On sait que $18\%$ des fran\c{c}ais souffrent d'allergie (plus ou moins grave) au gluten. 
On souhaite savoir si cette allergie peut être liée aux conditions de vie. On se demande, par exemple,
si les citadins sont différemment exposés. 
Pour ce faire, on a pris un échantillon de 250 individus vivant dans de grandes agglomérations
et on a répertorié 52 personnes allergiques au gluten. \\
Concluez cette étude en utilisant un test statistique. 
Donnez le degré de signification si nécessaire. \\

\bigskip 


\subsection*{ Exercice 3 : }
On souhaite tester l'efficacité d'un répulsif à moustiques proposé sur le marché. 
On considère un échantillon de 90 individus pris au hasard et on les expose à des moustiques pendant 30 minutes. On remarque que 63 de ces individus ont été piqués.
On considère ensuite un échantillon de 85 individus pris au hasard, on les asperge du répulsif anti-moustique, et on les expose à des moustiques pendant 30 minutes. On note alors 58 individus qui ont été piqués.

À l'aide d'un test statistique, dites si cette étude a mis en évidence l'efficacité du répulsif.
Précisez le degré de signification si nécessaire. \\

\bigskip 


\subsection*{ Exercice 4 : Pour s'entraîner}
Une étude a démontré que des composants de la réglisse stimulent la production de mucus par l’estomac et  renforcent ainsi, par exemple, sa protection naturelle contre les brûlures d'estomac. 

Des herboristes ont mis au point une tisane à base de réglisse pour empêcher l'apparition de problèmes de brûlure d'estomac et souhaitent connaître son efficacité. 
On note $\pi$ la proportion d'individus qui souffrent de brûlures d'estomac malgré la consommation
quotidienne de cette tisane.
Sur un échantillon de 85 individus qui ont consommé quotidiennement la tisane pendant six mois, on a comptabilisé 14 individus souffrant de brûlures d'estomac.

Donner une estimation de $\pi$ ainsi qu'un intervalle de confiance à $95\,\%$.\\

\bigskip 

\subsection*{ Exercice 5 : Pour s'entraîner}
Un laboratoire a trouvé une nouvelle molécules antipaludéenne pour empêcher les crises et souhaite en 
comparer l'efficacité avec une ancienne molécule.

On sait qu'avec l'ancienne molécule, $61\,\%$ des patients ne font plus de crise de paludisme. 

On a traité un échantillon de 367 patients avec la nouvelle molécule et on a observé que 235 n’ont 
plus fait de crise de paludisme par la suite.

Cette étude met-elle en évidence une différence d'efficacité entre les deux molécules ? 
On précisera le degré de signification si nécessaire. \\


\bigskip 

\subsection*{ Exercice 6 : Pour s'entraîner} 

\noindent La caféine peut contracter les vaisseaux sanguins cérébraux  et pourrait donc posséder des propriétés 
d'anti-migraineux. La question est de connaître la dose de caféine efficace. On note $\pi_1$ le taux
de guérison d'une migraine lorsqu'on absorbe 1 tasse de café et $\pi_3$ le taux
de guérison d'une migraine lorsqu'on absorbe 3 tasses de café. L'expérience suivante a été menée : 
\begin{itemize}
\item[-] un échantillon de 60 individus qui ont une migraine et on leur fait absorber une tasse de café ; on observe alors 21 cas de guérison dans l'heure qui suit.
\item[-] et un échantillon de 50 individus qui ont une migraine et on leur fait absorber trois tasses de café ; on observe alors 28 cas de guérison dans l'heure qui suit. 
\end{itemize} 
En utilisant un test statistique, dites si cette expérience a mis en évidence une différence d'efficacité 
entre les deux doses de caféines. \\
Si oui, quelle dose est la plus efficace ? Donnez le degré de signification, si nécessaire. \\

\bigskip 
\bigskip 
\bigskip 
\noindent {\bf Réponses des trois derniers exercices}.\\

\noindent Exercice 4 : 
On estime $\pi$ par 0.165. L'intervalle de confiance est $[0.086 ; 0.244]$.\\

\noindent Exercice 5 : On fait un test de conformité sur les proportions. 
On trouve $t\approx 1.19$. Il n'y a donc pas de différence significative.\\


\noindent Exercice 6 : On fait un test d'homogénéité sur les proportions.
On trouve $|t|\approx 2.21$ donc il y a une différence significative entre les deux taux de guérison.
La dose de trois cafés est plus efficace. Le degré de signification est environ 0.03.



\end{document}