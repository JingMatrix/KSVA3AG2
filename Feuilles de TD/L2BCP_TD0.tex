\documentclass[11pt]{article}
\usepackage{amsfonts,amssymb,epsfig}
%\usepackage[latin1]{inputenc}
\usepackage[all]{xy}
\usepackage{amsmath}
\usepackage{delarray}
\usepackage{pdfpages}
\usepackage[french]{babel}
\usepackage[T1]{fontenc}

\usepackage[utf8]{inputenc} 

\newtheorem{thm}{Theorem}[section]
\newtheorem{cor}[thm]{Corollary}
\newtheorem{lem}[thm]{Lemma}
\newtheorem{prop}[thm]{Proposition}
%\theoremstyle{definition}
\newtheorem{defn}[thm]{Definition}
%\theoremstyle{remark}
\newtheorem{rem}[thm]{Remark}
\numberwithin{equation}{section}
\newtheorem{example}[thm]{Example}

\newcommand{\D}{\displaystyle}
\newcommand{\N}{{\mathbb N}}
\newcommand{\Z}{{\mathbb Z}}
\newcommand{\K}{{\mathbb K}}
\newcommand{\R}{{\mathbb R}}
\newcommand{\C}{{\mathbb C}}

\setlength{\textwidth}{17.2cm}
\setlength{\textheight}{24cm}
\hoffset=-2.2cm
\voffset=-2.5cm

%-----------------------------------------------------------------------------
\begin{document}
\noindent Université Paul Sabatier -- Toulouse III
\hfill L2 BCP\\
\noindent Introduction aux Statistiques
%\hfill Année 2023-2024

\bigskip
\bigskip


\begin{center}
{\bf \large TD 0: Révisions sur les variables aléatoires discrètes}\\
{\bf \large }
\end{center}

\bigskip
%\bigskip

\subsection*{Exercice 1 :}

\noindent
Sur une île des Palaos, deux plages existent : l'une au nord et l'autre au sud. 
$1000$ touristes se trouvent sur cette île. Chaque jour, chaque touriste choisit d'aller sur l'une ou l'autre de ces deux plages, chacune  avec probabilité $1/2$, et ceci, indépendamment les uns des autres. On note $S$ la variable aléatoire égale au nombre de touristes choisissant la plage nord.
\begin{enumerate}
\item Quelle est la loi de $S$ ? Justifier votre réponse.
\item Donner l'espérance et la variance de $S$.
\item Calculer la probabilité qu'aucun touriste ne choisisse la plage nord.
\item Calculer la probabilité qu'au moins deux touristes choisissent la plage nord.
\item Les deux plages étant désertes au début de la journée, on dit qu'un touriste est \emph{heureux} si il se retrouve seul sur une plage toute  la journée. 
\begin{enumerate}
\item Peut-on avoir deux touristes heureux ?
\item
Montrer que la probabilité qu'il y ait un touriste heureux parmi les $1000$ touristes est
$ \displaystyle
\frac{1000}{2^{999}}.
$
\end{enumerate}
\end{enumerate}

\bigskip
\thispagestyle{empty}


\subsection*{Exercice 2 : }

\noindent
Certaines mutations sont à l'origine de plantes à feuille panachée. Si cela est souvent sélectionné 
chez les plantes d'ornement pour des raisons esthétiques, c'est peu souhaité pour les plantes 
agricoles, car les plantes sont plus fragiles et avec un rendement plus faible. 

On note $X$ la variable aléatoire représentant le nombre de plants panachés dans une parcelle d'un 
hectare de tournesol. Les mutations étant des phénomènes rares et plutôt spontanés, 
on suppose que $X$ suit une loi de Poisson.

\begin{enumerate}
\item Sur un échantillon de 300 parcelles de tournesol prises au hasard, on a remarqué qu'il y en 
avait 74 qui n'avaient aucun plant panaché. Donner une estimation du paramètre $\lambda$ de la loi
de Poisson que suit $X$.
\item Quel est le nombre moyen de plants panaché sur une parcelle de tournesol prise au hasard ?
\item Calculer la probabilité qu'une parcelle de tournesol possède au moins 3 plants panachés.
\end{enumerate}

\bigskip

\subsection*{Exercice 3 : }

Certaines graines ont besoin d'une période de froid pour pouvoir germer; c'est en particulier le cas 
pour certaines variétés de blé. On sait que, cultivées dans des conditions adaptées, avec une période de froid, $92\,\%$  
des graines d'une certaine variété de blé germent. 

On note $X$ la variable aléatoire représentant le nombre de graines qui ne germent pas parmi $50$ 
graines de cette variété semées dans de bonnes conditions de culture, avec une période de froid. 
\begin{enumerate}
\item Expliquer pourquoi $X$ suit une loi binomiale dont on précisera les paramètres. \\
Est-il possible d'approcher cette loi binomiale par une autre loi ? si oui laquelle ?
\item Quel est le nombre moyen de graines non germées ? et son écart-type ? 
\item Calculer les probabilités que : 
\begin{enumerate}
\item aucune graine n'ait germé, 
\item toutes les graines aient germé,
\item au moins une graine n'ait pas germé.  
\end{enumerate}
\end{enumerate}
\thispagestyle{empty}

\bigskip


\subsection*{Exercice 4 : }
En Europe, les moustiques vecteurs du paludisme sont rares et la transmission du 
parasite n’est pas systématique à chaque piqure. 
On désigne par Y la variable aléatoire qui
représente le nombre annuel de cas de paludisme en Europe. On supposera que le
nombre de cas observé dans une année est indépendant de ce qui a été observé les autres
années. On suppose que Y suit une loi de Poisson $\mathcal{P}(\lambda)$.

On sait que le nombre moyen annuel de cas de paludisme
en Europe depuis le milieu du XXième siècle est de 6 par an. 
%{\bf 1.} Rappeler la valeur de la moyenne $E(Y)$.
\begin{enumerate}
\item Déterminer une estimation de la valeur du paramètre $\lambda$.
\item Quelle est la probabilité que l’année prochaine, il y ait au moins 2 cas de paludisme en
Europe ?
\end{enumerate}



\end{document}
